\documentclass{article}
\usepackage[italian]{babel}
\usepackage[T1]{fontenc}
\usepackage[latin1]{inputenc}

\begin{document}
	
	\title{Monte Ventoso}
	\author{Lorenzo Bercelli \and Rita Maestri}
	\maketitle
	
	\section{Obiettivo}
	L'obiettivo del progetto \`{e} quello di simulare l'arrampicata di un individuo(scimmia) su una 
	parete liscia dotata di appigli, che costituiscono passaggi obbligati che la scimmia deve attraversare per poter arrivare in cima alla parete. 
	Per la creazione di una scimmia capace \`{e} stato scelto di usare un algoritmo genetico, in modo tale da far evolvere le strategie di salita della scimmia, ottimizzandole.
	\section{Piano di lavoro}
	\subsection{Primo tentativo: scimmie primitive}
	Il progetto prevede lo sviluppo di tre parti fondamentali:
	\begin{itemize}
	\item La creazione della parete su cui la scimmia si deve muovere: abbiamo ritenuto comodo schematizzare la parete come un grafo in cui gli appigli sono rappresentati dai nodi, i quali sono collegati tra loro solo se abbastanza vicini da permettere alla scimmia di passare da uno all'altro.
	\item La creazione della scimmia. Consiste nel decidere quali sono le possibili azioni della scimmia quando si trova davanti ad una certa configurazione degli appigli. Bisogna quindi determinare quanta parte di parete la scimmia ha facolt\`{a} di vedere e il set di possibili azioni che pu\`{o} intraprendere in base a ci\`{o} che vede attorno a s\'{e}. 

	\`{E} stato scelto, almeno nella fase iniziale del progetto, di attribuire alla scimmia una visuale e un set di possibili azioni ristretti: essa conosce infatti solo se tra i nodi che pu\`{o} raggiungere sono presenti:
	\begin{itemize}
		\item nodi che si trovano pi\`{u} in alto rispetto al punto in cui si trova e su cui \`{e} gi\'{a} passata;
		\item nodi che si trovano pi\`{u} in alto rispetto al punto in cui si trova e su cui non \`{e} mai passata;
		\item nodi che si trovano pi\`{u} in basso rispetto al punto in cui si trova e su cui \`{e} gi\'{a} passata;
		\item nodi che si trovano pi\`{u} in basso rispetto al punto in cui si trova e su cui non \`{e} mai passata.
	\end{itemize}

Quindi ci sono quattro categorie di nodi, ognuna delle quali \`{e} contrassegnata dall'attributo presente se il nodo su cui si trova la scimmia \`{e} collegato a un nodo della categoria considerata, o non presente in caso contrario.
Risulta quindi che ci sono $4^2$ configurazioni possibili nella visuale della scimmia.
Per ogni nodo in cui si trova, la scimmia deve scegliere a quale delle precedenti categorie deve appartenere il nodo su cui si spester\`{a}. Se sono presenti diversi nodi appartenenti alla categoria scelta lo spostamento avviene in modo casuale tra questi.

La strategia che ci aspettiamo segua una scimmia evoluta \`{e} di evitare di spostarsi su una categoria di nodi non presente tra quelli raggiungibili, di tendere a spostarsi sempre su nodi superiori mai visitati e nel caso il percorso intrapreso sia un vicolo cieco (non siano presenti nodi pi\`{u} in alto), si sposti in basso e intraprenda una strada alternativa (di nodi mai visitati).


 \item La creazione dell'algoritmo genetico: questo implica dotare le scimmia di un dna (che codifica la sua strategia), ovvero una funzione che specifica quale azione associare ad ogni possibile configurazione degli appigli, e di una funzione di fitness, che calcola la "bravura" della scimmia nello scalare la parete. L'algoritmo genetico vero e proprio usa questi due parametri per far evolvere le scimmie ed \`{e} cos\`{\i} strutturato: bisogna creare una prima generazione di scimmie dotate di strategie (dna) completamente casuali e usare la funzione di fitness per quantificare la bravura di tutte le scimmie della generazione. La generazione successiva sar\`{a} costituita da scimmie il cui dna \`{e} un incrocio tra i dna delle scimmie genitrici, con probabilit\`{a} di far ereditare i propri geni proporzionale alla propria bravura. Ci sar\`{a} inoltre una bassa probabilit\`{a} che alcune scimmie siano soggette a una mutazione casuale di un gene del loro dna.Si itera questo processo per un numero arbitrario di generazioni finch\`{e} la bravura delle scimmie non si stabilizza ad un punto di massimo: a questo punto \`{e} stat ottimizzata la strategia e presumibilmente le scimmie dell'ultima generazione sono capaci di arrivare in cima alla parete.
\end{itemize}


\subsubsection{Scimmia primititva sugli alberi}
Inizialmente abbiamo scelto di far muovere la scimmia per un determinato numero di passi attraverso un grafo ad albero, con un arbitrario numero di livelli e di figli per ciascun nodo, abbastanza semplice da poter essere attraversato anche con una visuale cos\`{\i} ristretta. \`{E} stata scelta una funzione di fitnness proporzionale alla somma delle altezze dei nodi visitati (quindi quanti pi\`{u} nodi in alto visita la scimmia tanto pi\`{u} \`{e} premiata) ma inversamente proporzionale al numero di nodi nella sua memoria, in modo tale che quelle arrivate in cima, avendo meno nodi in memoria, venivano premiate. \`{E} stato inoltre necessario prevenire la possibilit\`{a} che una scimmia facesse lo stesso movimento in loop (per esempio cercare di spostarsi su un nodo che non esiste, rimanendo quindi ferma) punendola nel caso lo facesse.

Il risultato \`{e} consistente con l'aspettativa: dopo meno di 100 generazioni la strategia delle scimmie si evolve in modo tale da farle arrivare, prima o poi, in cima. In particolare la scimmia esplora i nodi dell'albero con una ricerca depth first, quindi tendendo a spostarsi subito verso i nodi pi\'{u} alti e nel caso si trovi davanti a un vicolo cieco ritorna indietro ed esplora altre strade.

La pecca del modello utilizzato per queste scimmie primitive \`{e} che il percorso della scimmia \`{e} in buona parte affidato al caso, essendo casuale su quale dei nodi appartenenti alla categoria scelta la scimmia si muove. Quindi una stessa scimmia pu\`{o} impiegare un numero diverso di passi ad arrivare alla cima.
\subsubsection{Scimmia primitiva sulle pareti}
Questa pecca emerge non appena si prova a far muovere la scimmia su un grafo pi\`{u} complesso e pi\`{u} somigliante a una parete reale. Abbiamo generato questo tipo di grafo in questo modo: viene generata una distribuzione casuale di punti su un piano, che ha come parametri impostabili densit\`{a} dei punti, altitudine minima della cima (nodo con coordinata y maggiore), \`{a} di appiglio e probabilit\`{a} di appoggio. Tramite particolari accorgimenti \`{e} garantita la traversabilit\`{a} della parete cos\`{\i} generata. 

La scimmia che era capace di attraversare l'albero \`{e} solo in parte capace di attraversare la parete: se la parete ha uno sviluppo soprattutto verticale allora la scimmia \`{e} capace di attraversarla praticamente sempre, se invece la parete presenta strettoie e si sviluppa in orizzontale invece la scimmia pi\`{u} brava solo alcune volte arriva in cima: dipende dalla scelta casuale che fa dei nodi. Capita infatti pi\`{u} o meno la met\`{a} delle volte che vada in loop ai piedi della parete non trovando la direzione giusta da imboccare per sorpassare una strettoia.

Per far evolvere la scimmia sulle pareti sono state usate due diverse funzioni di fit, verificando quale fosse la migliore : quella usata anche per gli alberi ed un'altra, inversamente proporzionale al minimo numero di passi necessari per arrivare alla cima. Rsiulta che le scimmie che si sono evolute con la funzione di fit precedentemente usata hanno una strategia identica a quella che ci aspettiamo, quelle evolute con la seconda funzione arrivano in cima con la stessa probabilit\`{a} rispetto alle prime ma presentano una strategia leggermente diversa. Questo pu\`{o} essere determinato da una maggiore specializzazione della scimmia evoluta col secondo fit nei confronti della parete che si trova davanti.

\subsection{Buoni propositi}

\begin{itemize}
	\item Ampliare la visuale della scimmia, le possibilit\`{a} sono: 
	\begin{itemize}
		\item fare vedere alla scimmia anche a quale categoria appartengono i nodi raggiungibili in due passi;
		\item far vedere alla scimmia quante volte ha visitato un nodo per evitare il loop;
		\item dare una memoria geografica alla scimmia: sa quali nodi ha visitato e dove sono posizionati;
		\item indicare la direzione in cui si deve muovere con un vettore dato dalla combinazione lineare dei vettori che puntano ai nodi collegati con opportuni coefficienti dati dal dna;
	\end{itemize}
	
	\item implementare un'eventuale collaborazione tra scimmie: le scimmie possono raggiungere nodi che distano due collegamenti dal loro se in quei nodi sono presenti altre scimmie che le aiutano;
	\item analisi dati.
\end{itemize}

	
	
\end{document}