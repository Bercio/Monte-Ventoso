\documentclass{article}
\usepackage[italian]{babel}
\usepackage[T1]{fontenc}
\usepackage[utf8]{inputenc}

\begin{document}
	
	\title{Monte Ventoso}
	\author{Lorenzo Bercelli \and Rita Maestri}
	\maketitle
	
	\section{Obiettivo}
	L'obiettivo del progetto \`{e} quello di simulare l'arrampicata di un individuo(scimmia) fino al punto pi\`{u} alto di una parete liscia dotata di appigli, passaggi obbligati.
	\section{Diario di bordo}
	\subsection{Primo tentativo: scimmie primitive}
	Il progetto prevede lo sviluppo di quattro parti fondamentali.
        \subsubsection{La creazione della parete} Abbiamo modellato la parete come un grafo in cui gli appigli sono rappresentati come nodi a cui abbiamo assegnato coordinate cartesiane, i quali sono connessi solo se abbastanza vicini da permettere alla scimmia di passare da uno all'altro. Abbiamo inizialmente testato le scimmie su degli alberi, per poi passare a pareti pi\`{u} complesse: grafi direzionati. Specificheremo in seguito le caratteristiche dei due tipi di pareti.
	
        \subsubsection{La creazione della scimmia} Abbiamo determinato quanta parte di parete la scimmia ha facolt\`{a} di vedere e il set di possibili azioni che pu\`{o} intraprendere in base a ci\`{o} che vede attorno a s\'{e}, cio\`{e} la sua strategia.

	Abbiamo scelto, almeno nella fase iniziale del progetto, di attribuire alla scimmia una visuale e un set di possibili azioni ristrette: essa sa solo se i nodi che pu\`{o} raggiungere sono:
	\begin{itemize}
		\item nodi che si trovano pi\`{u} in alto rispetto al punto in cui si trova e su cui \`{e} gi\'{a} passata;
		\item nodi che si trovano pi\`{u} in alto rispetto al punto in cui si trova e su cui non \`{e} mai passata;
		\item nodi che si trovano pi\`{u} in basso rispetto al punto in cui si trova e su cui \`{e} gi\'{a} passata;
		\item nodi che si trovano pi\`{u} in basso rispetto al punto in cui si trova e su cui non \`{e} mai passata.
	\end{itemize}

Risulta quindi che ci sono $2^4$ possibili visuali della scimmia.
Per ogni spostamento la scimmia deve scegliere a quale delle precedenti categorie deve appartenere il nodo su cui si sposter\`{a}. Se sono presenti diversi nodi appartenenti alla categoria scelta lo spostamento avviene in modo casuale tra questi. Se non sono presenti nodi connessi appartenenti alla categoria scelta la scimmia resta ferma.

La strategia che ci aspettiamo segua una scimmia evoluta \`{e} di evitare di spostarsi su una categoria di nodi non presente tra quelli raggiungibili, di spostarsi sempre su nodi superiori mai visitati e nel caso il percorso intrapreso sia un vicolo cieco (non siano presenti nodi raggiungibili pi\`{u} in alto), si sposti in basso e intraprenda una strada alternativa (cio`{e} scelga nodi mai visitati).

\subsubsection{La creazione dell'algoritmo genetico} Abbiamo dotato la scimmia di un dna che codifica la sua strategia, ovvero un'applicazione lineare che specifica quale azione associare ad ogni possibile visuale della scimmia, e di una funzione di fitness che calcola la "bravura" della scimmia nello scalare la parete. Abbiamo definito il valore di fitness come proporzionale alla somma delle altezze dei nodi visitati -- quanti pi\`{u} nodi in alto visita la scimmia tanto pi\`{u} \`{e} premiata -- ma inversamente proporzionale al numero di spostamenti effettuati. Poich\`{e} il numero di spostamenti massimi che la scimmia compie prima di valutare il suo fitness \`{e} una costante, e una scimmia smette di spostarsi appena raggiunge la cima della parete, questa componente del fitness premia la velocit\`{a} con cui la scimmia raggiunge la cima. \`{E} stato inoltre necessario punire le scimmie che facessero lo stesso movimento in loop (per esempio cercare di spostarsi su un nodo che non esiste, rimanendo quindi sullo stesso nodo) sottraendo loro una costante dal fitness.
 
L'algoritmo genetico usa questi due parametri,strategia e fitness, per far evolvere le scimmie: partendo da una prima generazione di scimmie dotate di strategie (dna) completamente casuali determina il loro fitness dopo aver applicato le loro strategie all'attraversamento della parete. La generazione successiva \`{e} costituita da scimmie il cui dna \`{e} un incrocio tra i dna delle scimmie della generazione precedente, con probabilit\`{a} di far ereditare i propri geni proporzionale al proprio fitness. C'\`{e}  inoltre una probabilit\`{a} costante che alcune scimmie siano soggette a una mutazione casuale di un elemento del loro dna. L'algoritmo ripete quest'ultimo processo un numero arbitrario di volte da noi impostato.

\subsubsection{Scimmia primitiva sugli alberi}
Inizialmente abbiamo scelto di far muovere la scimmia per un determinato numero di passi attraverso un grafo ad albero, con un arbitrario numero di livelli e di figli per ciascun nodo, abbastanza semplice da poter essere attraversato anche con una visuale cos\`{\i} ristretta.
Il risultato \`{e} consistente con l'aspettativa: dopo meno di 100 generazioni la strategia delle scimmie si evolve in modo tale da farle arrivare, prima o poi, in cima. In particolare la scimmia tende ad adottare una strategia d'esplorazione \textit{depth first}.

La pecca del modello utilizzato per queste scimmie primitive \`{e} che il percorso della scimmia \`{e} in buona parte affidato al caso, essendo casuale su quale dei nodi appartenenti alla categoria scelta la scimmia si muove. Quindi una stessa scimmia pu\`{o} impiegare un numero diverso di passi ad arrivare alla cima. Inoltre sono pochissime le informazioni su cui la scimmia pu\`{o} basare la sua scelta.

\subsubsection{Scimmia primitiva sulle pareti}

I limiti del modello si manifestano non appena si prova a far muovere la scimmia su un grafo pi\`{u} complesso e pi\`{u} somigliante a una parete reale. Per creare questo tipo di grafo innanzitutto generiamo una distribuzione casuale di punti su un rettangolo e connettiamo direzionalmente tra loro i punti che distano meno di una costante arbitraria. Assegniamo l'orientamento degli archi seguendo una distribuzione casuale definita da parametri costanti che determinano la probabilit\`{a} che un arco connetta un punto pi\`{u} alto ad un punto pi\`{u} basso, viceversa, o che sia bidirezionale (queste direzioni modellano un appiglio unicamente utilizzabile per appoggiarsi o per trarsi). Tra i grafi debolmente connessi generati da questo procedimento filtriamo i grafi in cui il nodo pi\`{u} basso e nodo pi\`{u} alto sono fortemente connessi e che contengano un albero di profondit\`{a} superiore ad una costante arbitraria (al fine di ottenere una parete sufficientemente difficile da attraversare). Tra questi selezioniamo quello con cardinalit\`{a} maggiore.

La scimmia che era capace di attraversare l'albero \`{e} solo in parte capace di attraversare la parete: se la parete ha uno sviluppo soprattutto verticale allora la scimmia \`{e} capace di attraversarla praticamente sempre, se invece la parete presenta strettoie e si sviluppa in orizzontale la scimmia pi\`{u} brava solo alcune volte arriva in cima: dipende dalla scelta casuale che fa dei nodi. Capita infatti pi\`{u} o meno la met\`{a} delle volte che vada in loop ai piedi della parete non trovando la direzione giusta da imboccare per sorpassare una strettoia.

Per far evolvere la scimmia sulle pareti sono state usate due diverse funzioni di fitness, verificando quale fosse la migliore: quella usata anche per gli alberi ed un'altra,che definisce il valore di fitness come inversamente proporzionale al minimo numero di passi necessari per arrivare alla cima. Risulta che le scimmie che si sono evolute con la funzione di fitness precedentemente usata hanno una strategia quasi identica a quella che ci aspettiamo, quelle evolute con la seconda funzione arrivano in cima con la stessa probabilit\`{a} rispetto alle prime ma presentano una strategia leggermente diversa. Questo pu\`{o} essere dovuto a una maggiore specializzazione della scimmia evoluta col secondo fit nei confronti della parete che si trova davanti.

\subsection{Modifica della visuale della scimmia: correzione del loop}
Per risolvere il problema del loop abbiamo adottato una soluzione ad hoc: abbiamo ampliato il numero di informazioni disponibili alla scimmia aggiungendo il caso in cui si trovasse in un loop.
Quindi le configurazini possibili, nonch\`{e} il numero di elementi nel vettore di dna, sono diventati $2^5=32$. 

Dopo questa modifica l'algoritmo \`{e} diventato pi\`{u} efficace: compiuto un numero sufficiente di evoluzioni, la scimmia \`{e} sempre capace di attraversare la parete su cui predentemente aveva avuto difficolt\`{a} a causa del loop.

Ma testando la scimmia su qualche decina di pareti random  abbiamo notato che non sempre la strategia adottata le permette di raggiungere la cima. In particolare la scimmia fallisce circa 1 volta su 5. Questo implica che la soluzione trovata corregge solo un comportamento sbagliato ma non permette di elaborare una strategia universalmente valida.

\end{document}
